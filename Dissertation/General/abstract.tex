\chapter*{Abstract}
\vspace*{-1cm}
Automatic Speech Recognition (ASR) functionality,
the automatic translation of
speech into text, 
is on the rise today and is required for 
various use-cases,
scenarios, and applications.
However, 
an ASR engine by itself will face difficulties 
performing decently, 
regardless of how sophisticated and advanced it may be.
That is true, 
especially under the circumstances 
such as having a noisy ambient environment,
multiple speakers, or faulty microphones.
These kinds of challenges characterize 
a realistic scenario for an ASR system.

In time, the idea of ASR functionality evolved
to a more comprehensive, End-to-End (E2E) solution.
One that is robust, to some extent, to 
external interferences, 
flexible, so that it can be extended 
to adapt to new scenarios or for performance increase,
and modular enough to 
conveniently reform and be compatible 
with new applications.
Such an E2E ASR solution may include 
several additional micro-modules 
of speech enhancements besides the ASR engine, 
which is very complicated by itself. 
Adding these micro-modules can enhance the robustness 
and improve the overall system's performance.
Examples of such possible micro-modules include 
noise cancelation and speech separation, 
multi-microphone arrays, and adaptive beamformer(s).
Being a comprehensive solution built of
numerous micro-modules is technologically 
challenging to implement
and challenging to integrate into resource-limited
mobile systems. 
By offloading the complex computations to a server
on the cloud, 
the system can fit more easily in less capable computing devices. 
Nevertheless, that compute offloading comes with the cost of
giving up on real-time analysis. 
In addition, offloading to a server must have
connectivity to the cloud over the internet.

To find the optimal trade-offs between performance,
Hardware (HW) and Software (SW) 
requirements or limitations, 
maximal computation time 
allowed for real-time analysis,
and the detection accuracy,
one should first define the different metrics 
used for the evaluation 
of such an E2E ASR system.
Secondly, one needs to determine 
the extent of correlation between those 
metrics, plus the ability to forecast the
impact each variation has on the others.

This research presents novel progress in optimally designing 
a robust E2E-ASR system
targeted for mobile, 
resource-limited devices. 
First, we describe evaluation metrics for each domain of interest, 
spread over vast engineering subjects. 
Then, we emphasize any bindings between the metrics 
across domains and the degree of impact 
derived from a change in the system's specifications or constraints. 
Second, we present the effectiveness
of applying machine learning techniques 
that can generalize and provide results 
of improved overall performance and robustness.
Third, we present an approach of substituting
architectures, changing algorithms,
and approximating complex computations
by utilizing 
a custom dedicated hardware acceleration 
in order to replace the 
traditional state-of-the-art SW-based 
solutions, thus providing real-time 
analysis capabilities to resource-limited systems.

