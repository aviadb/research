\chapter{Conclusions}\label{ch:concl_ch}
\vspace{-1cm}
In this research we have 
investigated some possible 
explicitly or implicitly direct links
between
several metrics that are measured 
in different stages along an E2E ASR system.

\bigskip

We also investigated the 
impact of speech enhancement 
where we have seen how
a multi-channel microphone array beamformer
combined with a T-F masking algorithm
can improve an existing E2E ASR system detection rates.
This provided insights into a correlation 
between improvements in SNR, PESQ, and STOI
leading 
to an improvement in WER and CER whom
are ASR metrics.

\bigskip

We also presented
the magnitude of influence
that different scaling methods have 
on ASR system's performance and requirements,
especially when the implementation is targeted for
a hardware device rather than a software based solution.
In such cases, the requirements for
computation efficiency, low power consumption
and wise utilization of limited resources
are taken into account and affect the 
ranks that different configuration sets
get. For example, slim models for the
beamformer, masking or the ASR engine 
would be ranked higher than heavy and 
larger models
when the target device has limited memory resources.
Alternatively, when storage memory is not considered an issue, 
a better performing algorithm may 
be achieved using additional memory. 
Such an algorithm would yield higher detection rates.
% Alternatively, when storage memory is
% not considered an issue, a better performing
% algorithm would be more preferable
% for higher detection rates.
\bigskip

To evaluate the End-to-End speech enhancement
plus the ASR engine performance, we divided
the effort into two sections, where each has been
evaluated separately. Later, the two sections 
were connected together and the evaluation was
taken from input to the ASR output.

\bigskip

During experiments,
multiple ASR engines were trained against
different carefully selected feature sets,
each time alternating one parameter,
whether the features' combination,
scaling method, the computation for feature extraction
or the neural network's shape and structure.
A clear advantage, 
in terms of ASR performance (WER, CER), 
has been observed for a combination of the
Delta, Deltas-Delta \((\Delta, \Delta\Delta)\) 
and the companion
cepstral coefficients, over the increase in the
number of filters and coefficients, 
or the usage of only FilterBanks.
Adding temporal context to the mixture,
resulted in a slightly better ASR performance.
These insights reflected mainly in 
the effort of simplifying our ASR model
and reducing its memory footprints.
In accordance, simplifications such as
approximations for the scaling methods
and the reduction in filters and coefficients
were made possible while maintaining
equal detection rates and sometimes
even better than highly dense
FilterBank based ASR systems.
On the other hand, root cepstral coefficients
extraction did not yield any gain in performance
over the natural or log based coefficients, nor
the suggested approximations.

\bigskip

Those trained ASR engines were attached to
a processing FE (Front-End), that serves as
a speech enhancing stage. The speech enhancing
is applied by multi channel multi microphones array
deep beamformers combined with T-F maskins.
That combination between an GEV beamformer 
and a T-F masking extraction later attached to
an ASR engine led to much improved
performance compared to the ASR performance
measured as a standalone unit.
Deeper analysis of the various T-F masking alogirthms
yielded a conclusion that the audio domain metrics
such as SNR and STOI have direct impact on 
the ability of the ASR engine to be more precise.
This actually strengthen the intuition that 
better speech quality or cleaner speech
is easier to perceive, and thus 
ASR transcripts would be more accurate for it.

\bigskip

Some very important questions arise with regard to 
adding a FE stage in front of the ASR engine,
that is what would be the cost? What could be traded
in favor of performance or higher detection ratios?
In this research we used the same configuration
for the feature extractions in both the
FE and the ASR engine itself. 
Doing so, we manage to combine the feature extraction
mechanism and save on resources by reusing
the same modules. The computation overload
for the FE part becomes negligible since it is 
already implemented for the ASR, thus
only the extraction of the beamformer coefficients is
what left for computation. Despite saving on resources,
an increase in memory is anticipated to hold
the beamformer's and the T-F masking NN model parameters.

\bigskip

Also presented in this research trade-off options,
all together with the expected gains in some metrics
side by side with the improvements in WER and CER.
For systems that are not designed with limited resources,
like memory and slow operating clock frequency, 
the more advanced, but complex algorithms for T-F masking
as cIRM, PSM, and ORM can be used.
Leading to higher perception rates at the ASR output
as mentioned before due to higher SNR levels. However,
trading-off complex T-F masking with the simpler and
lighter IRM implementation while selecting more optimized
ASR engine and feature set can yield quietly matched
WER and CER results. According to the results presented
in this research, speech to text systems can be optimized
by wise selection 
of configurations for certain
modules and yet achieving desired quality of performance.


\section{Future work}
In this research we based our ASR engine
architecture on the transformer model.
Other advanced ASR engine models are available
such as CTC based Seq2Seq and the transducer model.
Also, modern techniques of introducing 
additional acoustic information to the ASR training process
like the wav2vec outputs are expected to improve the results presented
in this research. 

Besides the ASR, further optimizations in hardware
can be applied with dedicated HW based accelerators,
which are now built-in inside new AI capable FPGA devices.
These hardware accelerators are tailor made for
neural networks, mainly CNNs. 
The systolic-array architecture that is very common
for such accelerators provide much lesser computing times
when compared to conventional piped-line hardware architectures.
Furthermore, the computing efficiency measured in
tera-operations per energy or power units
\([\sfrac{TFLOPS}{Joule}, \sfrac{TFLOPS}{Watt}]\) of such accelerators
is tens to hundreds times higher.

Going forward, combining both
the alternative ASR engine models 
and more expanded optimizations in hardware,
this research can be more comprehensive in terms of
the possibilities laid in front of a system designer
for taking trade-offs based on the system's requirements.
