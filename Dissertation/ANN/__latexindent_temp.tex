
% \chapter{ANN - Artificial Neural Networks}
% \section{Basic Architecture}

\chapter{DNN - Deep Neural Networks}
\section{ANN - Artificial Neural Networks}
\section{Basic Architecture}

\section{NN Layers}
\subsection{Linear Layers}
\subsubsection{Linear}
\subsubsection{Bi-Linear}

\subsection{Convolution Layers}
\subsubsection{Convoltion operation}
\begin{equation} \label{eq:conv_operation}
    \text{out}(N_i, C_{\text{out}_j}) = \text{bias}(C_{\text{out}_j}) +
    \sum_{k = 0}^{C_{in} - 1} \text{weight}(C_{\text{out}_j}, k)
    \star \text{input}(N_i, k)  
\end{equation}

Both the weights and bias parameters are learnable during the training
process of the network.

\subsubsection{Convolutional Layer Parameters}
\subsection*{Kernel}
A kernel, or a filter, is a set of values that convolves with 
the input feature map
as part of the convolutional neural network. 
The convolution layer takes the kernel as one operand,
and ``slides'' it over the second operand. 
As seen in Equation \ref{eq:conv_operation}, 
the convolution operation, sums the dot product of two operands
over the total number of weight values in the kernel. 

\begin{figure}[H]
    \centering
    \begin{tikzpicture}[scale=.5,every node/.style={minimum size=1cm}, on grid]
            \draw[fill=base02a,opacity=0.4] (0,0) rectangle (3,3);
            \draw[draw=base03,thick] (0,0) grid (3,3);
            \node (00) at (0.5,2.5) {\tiny 0};
            \node (01) at (1.5,2.5) {\tiny 1};
            \node (02) at (2.5,2.5) {\tiny 2};
            \node (10) at (0.5,1.5) {\tiny 3};
            \node (11) at (1.5,1.5) {\tiny 4};
            \node (12) at (2.5,1.5) {\tiny 5};
            \node (20) at (0.5,0.5) {\tiny 6};
            \node (21) at (1.5,0.5) {\tiny 7};
            \node (22) at (2.5,0.5) {\tiny 8};
    \end{tikzpicture}
\end{figure}

The kernel above, is of size \( \left( \left[ 3, 3 \right] \right) \).
Presenting it as a 4D-Tensor, would give: \( \left( N, C, H, W \right)  = \left( \left[ 1, 1, 3, 3 \right] \right)  \)
% \begin{align*}
    
% \end{align*}

\subsection*{Stride}
The stride parameter controls the steps in which the kernel 
moves across the input feature map. 
It takes two values, \( \left( \left[ H_{s}, W_{s} \right] \right) \), 
where \( H_{s}, W_{s} \) are the integer step values for the 
height and the width movements across the input feature map.

\begin{figure}[H]
    \centering
\begin{tikzpicture}[scale=.5,every node/.style={minimum size=1cm}, on grid]
    \begin{scope}[xshift=0,yshift=0cm]
        \begin{scope}[xshift=0cm,yshift=0cm]
            \draw[draw=base03,fill=blue,thick] (0,0) grid (5,5) rectangle (0,0);
            \draw[fill=base02a, opacity=0.4] (0,2) rectangle (3,5);
        \end{scope}
        % \begin{scope}[xshift=7cm,yshift=1.5cm]
        %     \draw[draw=base03,fill=cyan,thick] (0,0) grid (2,2) rectangle (0,0);
        % \end{scope}
    \end{scope}
    \draw[draw=base03, ->, thick] (2.6,3.5) to  (4.5,3.5);
    \draw[draw=base03, ->, thick] (1.5,2.4) to (1.5,0.5);
    % \draw[draw=base03, ->, thick] (5.25, 2.5) to (6.75, 2.5);
    % \begin{scope}[xshift=12cm,yshift=0cm]
    %     \begin{scope}[xshift=0cm,yshift=0cm]
    %         \draw[draw=base03,fill=blue,thick] (0,0) grid (5,5) rectangle (0,0);
    %         \draw[fill=base02, opacity=0.4] (0,2) rectangle (3,5);
    %     \end{scope}
    %     \begin{scope}[xshift=7cm,yshift=1cm]
    %         \draw[draw=base03,fill=cyan,thick] (0,0) grid (3,3) rectangle (0,0);
    %         \draw[draw=base03] (1,0) -- (2,1) -- (2,0) -- (1,1);
    %         \draw[draw=base03] (0,1) -- (1,2) -- (1,1) -- (0,2);
    %         \draw[draw=base03] (1,1) -- (2,2) -- (2,1) -- (1,2);
    %         \draw[draw=base03] (2,1) -- (3,2) -- (3,1) -- (2,2);
    %         \draw[draw=base03] (1,2) -- (2,3) -- (2,2) -- (1,3);
    %     \end{scope}
    %     \begin{scope}[xshift=12cm,yshift=1.5cm]
    %         \draw[draw=base03,fill=cyan,thick] (0,0) grid (2,2) rectangle (0,0);
    %     \end{scope}
    % \end{scope}
    % \draw[draw=base03, ->, thick] (14.6,3.5) to  (15.5,3.5);
    % \draw[draw=base03, ->, thick] (15.6,3.5) to  (16.5,3.5);
    % \draw[draw=base03, ->, thick] (13.5,2.4) to (13.5,1.5);
    % \draw[draw=base03, ->, thick] (13.5,1.4) to (13.5,0.5);
    % \draw[draw=base03, ->, thick] (17.25, 2.5) to (18.75, 2.5);
    % \draw[draw=base03, ->, thick] (22.25, 2.5) to (23.75, 2.5);
\end{tikzpicture}
\end{figure}

\subsection*{Padding}
Padding is a method of extending the input feature map with 
zero values all around in order to control the size dimensions of the 
output feature map. This parameter takes 
\begin{figure}[H]
    \centering
\begin{tikzpicture}[scale=.5,every node/.style={minimum size=1cm}, on grid]
    \draw[draw=base03, ->, thick] (1.5,2.4) to (1.5,0.5);
\end{tikzpicture}
\end{figure}

\subsection*{Dilation}
dilation

\subsubsection{Conv Layer Output}
\begin{align}
    H_{out} =& \left\lfloor\frac{H_{in}  + 2 \times \text{padding}[0] - \text{dilation}[0]
                        \times (\text{kernel\_size}[0] - 1) - 1}{\text{stride}[0]} + 1\right\rfloor  \\
    W_{out} =& \left\lfloor\frac{W_{in}  + 2 \times \text{padding}[1] - \text{dilation}[1]
                        \times (\text{kernel\_size}[1] - 1) - 1}{\text{stride}[1]} + 1\right\rfloor
\end{align}


\subsection{Recurrent Layers}
\subsubsection{LSTM}
\subsubsection{BLSTM}
\subsubsection{RNN}

\subsection{Transformer Layers}
\subsubsection{Transformer}
\subsubsection{Transformer-Encoder}
\subsubsection{Transformer-Decoder}


\section{NN Function Layers}
\subsection{Non-linear Activations Functions}
\subsubsection{ReLU}
\subsubsection{LeakyReLU}
\subsubsection{Sigmoid}
\subsubsection{Tanh}
\subsubsection{Softplus}

\subsection{Normalization Functions}
\subsubsection{Batch-Normalization}

\subsection{Pooling Functions}
\subsubsection{Min-pooling}
\subsubsection{Average-pooling}
\subsubsection{Max-pooling}

\section{Dropout Layers}
\subsubsection{p-Dropout}
To avoid some redundant feature(s) detection by multiple neurons in a network,
a co-adaptation amongst the different neurons in the model should be prevented. 
In that way, the NN model is much more effective and uses resources more efficiently.

This holds true especially during training, as described in this paper
\citep{hinton2012improving}.

The idea is to zero-out, arbitrarily, values of
the input feature. 
Based on the \emph{Bernoulli distribution}, the probability to zero an element gets the
probability \(p\), while the opposite, non-zeroing probability is set to \(1-p\). 

Raising the \(p\) value too high, may lead to an exhaustive training process, thus 
a good balance point should be used to overcome the co-adaptation while not missing 
useful features. Whenever the training sets are considered large, a small portion 
of zeroed elements should be sufficient for satisfactory results.  

\subsection{Loss Functions}
\subsubsection{L1 Loss (MAE)}
\subsubsection{L2 Loss (MSE)}
\subsubsection{Cross-Entropy}
\subsubsection{CTC}